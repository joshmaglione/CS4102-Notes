\documentclass[a4paper]{amsart}

\usepackage{enumerate}
\usepackage{hyperref}
\hypersetup{
	colorlinks=true,
	linkcolor=blue,
	filecolor=blue,
	urlcolor=blue,
	citecolor=blue,
}
\usepackage{amsmath}
\usepackage{amsthm}
\usepackage{amssymb}
\usepackage[margin=3cm]{geometry}
\usepackage{mathpazo}
\usepackage{url}
\usepackage[labelformat=simple]{subcaption}
\usepackage{tikz}
\usepackage{pgf}
\usepackage{longtable}
\usepackage{multirow}
\usepackage{graphicx}
\usepackage{pgfplots}
\usepackage{cleveref}
\usepackage{amsrefs}
\usepackage{bbm}

\numberwithin{equation}{section}
\numberwithin{figure}{section}

\newtheorem{thm}{Theorem}[section]
\newtheorem*{thm*}{Theorem}
\newtheorem*{con*}{Conjecture}
\newtheorem{lem}[thm]{Lemma}
\newtheorem{prop}[thm]{Proposition}
\newtheorem{cor}[thm]{Corollary}
\newtheorem{lemma}[thm]{Lemma}
\newtheorem{conj}[thm]{Conjecture}

\theoremstyle{definition}
\newtheorem{defn}[thm]{Definition}
\newtheorem{remark}[thm]{Remark}
\newtheorem{ex}[thm]{Example}
\newtheorem{quest}[thm]{Question}
\newtheorem{obs}[thm]{Observation}
\newtheorem{prob}{Problem}

\renewcommand{\leq}{\leqslant}
\renewcommand{\geq}{\geqslant}
\newcommand{\N}{\mathbb{N}}
\newcommand{\Z}{\mathbb{Z}}
\newcommand{\Q}{\mathbb{Q}}
\newcommand{\R}{\mathbb{R}}
\newcommand{\C}{\mathbb{C}}
\newcommand{\tr}{\mathrm{t}}

\title{Problem Set 1}
\date{}

\begin{document}

\maketitle

\section*{Instructions}

Three problem sets count for 40\% of the module assessment, and the exam counts
for the other~60\%. The lowest of the three sets will be ignored.

Each homework should be submitted Canvas as an archive file (zip or tar)
consisting of
\begin{enumerate}
  \item your python code (a \texttt{.py} file or a Jupyter Notebook),
  \item a pdf version of a written report,
  \item any data file used as input for your python program.
\end{enumerate}
The \texttt{.py} file (or Jupyter Notebook) should be a machine readable version
of the appendix code which, when run, reproduces your answers. Develop your own
Python code rather than simply using existing Python modules for linear
regression (e.g.\ \texttt{statsmodels}). The \texttt{.pdf} document should
contain your answers to the questions in which you provide a description of the
mathematical methods used. There should be no Python code. 

The homework will be graded according to a scheme in which {\em content} (i.e.\
correctness of your answers, choice of methods, python code) is weighted at 70\%
and {\em presentation} (i.e.\ manner in which you present your answers, methods
and code) is weighted at 30\%.

Please submit on Canvas by \textbf{09.10.2023} as a single archive file
(tar or zip or similar).

\vspace{1em}

\hrule

\vspace{1em}

\begin{prob}

\end{prob}


\subsection{}
The scatter plot
% $$\includegraphics[height=6cm]{data1.eps}$$
represents a set of points
$(x_1,y_1), (x_2,y_2), \ldots, (x_{100},y_{100})$ produced using a
model $y_i = \beta_0 +\beta_1x_i +\beta_2x_i^2 +\epsilon_i$ with
independent random errors $\epsilon_i$ of mean $0$ and finite
variance.  The numerical values of the points $(x_i,y_i)$
are contained in the file \verb!data1.txt! where the first $9$
lines look as
follows:

% \verbatiminput{data0.txt}


\begin{enumerate}
\item Determine the values of $b_0$, $b_1$, $b_2$ for which
$$y = b_0 + b_1x + b_2 x^2$$
is the least squares estimator for the model  $y_i = \beta_0 +\beta_1x_i +\beta_2x_i^2 +\epsilon_i$.

\item Exhibit a single plot of the data points (in say blue)
 and the curve $y = b_0 + b_1x + b_2 x^2$ (in say red).

\item Determine the coefficient of determination $r^2=1-(SSE/SSTO)$ for this least squares fit.
\end{enumerate}

\subsection{}
The observations below, taken on $10$ incoming shipments of chemicals in drums arriving at a warehouse, show number of drums in shipment ($x_1$), total weight of shipment ($x_2$, in hundred pounds), and number of man-minutes required to handle the shipment ($y_i$):
$$\begin{array}{r|llllllllll}
 i:&1&2&3&4&5&6&7&8&9&10\\
\hline x_{i1}: & 7 &18 &5 &14 &11 &5 &23 &9&16 &5\\
x_{i2}: & 5.11 &16.70 &3.20 &7.00 &11.00 &4.00 &22.10 &7.00 &10.60 &4.80\\
y_i: &58 &152 &41 &93 &101 &38 &203 &78 &117 &44
\end{array}$$

\begin{enumerate}
\item Assume a model
\begin{equation}
y_i = \beta_0+\beta_1x_{i1} + \beta_2 x_{i2} +\epsilon_i
\label{eq1}
\end{equation}
in which errors are independent $N(0,\sigma^2)$.
\begin{enumerate}

\item Determine the least squares estimator $y=b_0+b_1x_1 + b_2x_2$.

\item Test whether there is a regression equation, using a level of significance of 0.05.

\item Estimate $\beta_1$ and $\beta_2$ jointly, using a 95\% family confidence coefficient.

\item Management desires simultaneous interval estimates of the mean handling times for five typical shipments specified to be as follows:
$$\begin{array}{r|lllll}
&1 &2 &3 &4 &5\\
\hline x_1: &5 &6 &10 &14 &20\\
x_2: &3.20 &4.80 &7.00 &10.00 &18.00
\end{array}$$
Obtain the family of estimates, using a 90\% family confidence coefficient.
\end{enumerate}


\item Obtain the residuals and make appropriate residual plots to ascertain whether model (\ref{eq1}) with normal error terms is appropriate. Summarize your findings.
\end{enumerate}



\end{document}
