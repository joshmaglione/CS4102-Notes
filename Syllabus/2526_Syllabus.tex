\documentclass[a4paper, 12pt]{article}

\usepackage{enumerate}
\usepackage{hyperref}
\hypersetup{
	colorlinks=true,
	linkcolor=blue,
	filecolor=blue,
	urlcolor=blue,
	citecolor=blue,
}
\usepackage{amsmath}
\usepackage{amsthm}
\usepackage{amssymb}
\usepackage[margin=3cm]{geometry}
\usepackage{mathpazo}
\usepackage{url}
\usepackage[labelformat=simple]{subcaption}
\usepackage{tikz}
\usepackage{pgf}
\usepackage{longtable}
\usepackage{multirow}
\usepackage{graphicx}

\begin{document}
\pagestyle{empty}

\begin{center}
{\Large Geometric Foundations of Data Analysis I (CS4102)} 

\vspace{0.25cm}

{\large Joshua Maglione}

\vspace{0.25cm}

Semester 1 (2025)
\end{center}

\vspace{0.5cm}

\begin{description}
    \item[Module information:] \hfill
    \begin{description}
      \item[Meeting Times:] \hfill
      \begin{center}
        \begin{tabular}{rl}
          Mondays & 3:00pm -- 3:50pm \\ 
          Wednesdays & 10:00am -- 10:50am
        \end{tabular}
      \end{center}
      \item[Room:] \'Aras de Br\'un 1020,
      \item[Contact:] \url{joshua.maglione@universityofgalway.ie} 
      \item[Website:] \href{https://universityofgalway.instructure.com/}{\textsf{Canvas}} and \url{https://joshmaglione.com/2025CS4102.html} 
    \end{description} 
    \vspace{1cm}
    \item[Topics:] We will cover four key methods of data analysis:
    \begin{enumerate} 
      \item Least Squares Fitting,
      \item Principal Component Analysis,
      \item Hierarchical Clustering and Persistence,
      \item Nearest Neighbors and the Johnson--Lindenstrauss Theorem
    \end{enumerate}
    We will also explore these topics in Python using Jupyter Notebooks.
    \vspace{1cm}
    \item[Assessment:] The total assessment of the module comprises three
    components: four in-class quizzes (20\%), an in-class exam (40\%), and a
    final exam (40\%). We will have a quiz every other week, for a total of six
    quizzes. The two lowest scores will not be counted. The in-class exam is
    tentatively scheduled for the 22nd of October (Wednesday in week 7);
    confirmed by the second week.
    \vspace{1cm}
    \item[Reading list:] The following list is not required but could be useful.
    \begin{enumerate} 
      \item Blum, Avrim; Hopcroft, John; and Kannan Ravindran.
      \textit{Foundations of Data Science}. 2018.
      \url{https://www.cs.cornell.edu/jeh/book.pdf}.
      \item Hastie, Trevor; Tibsshirani, Robert, and Friedman, Jerome. \textit{The Elements of Statistical Learning}. Second edition -- 12th printing, Springer Ser.\ Statist.\ Springer, New York, 2009. \url{https://hastie.su.domains/ElemStatLearn/printings/ESLII_print12_toc.pdf}.
      \item Jolliffe, I. T. \textit{Principal Component Analysis}. Second
      edition, Springer Ser.\ Statist.\ Springer-Verlag, New York, 2002.
      \item Margalit, Dan and Rabinoff, Joseph. \textit{Interactive Linear
      Algebra}. 2019. \url{https://textbooks.math.gatech.edu/ila/}.
      \item Phillips, Jeff M. \textit{Mathematical Foundations for Data Analysis}. Springer Ser.\ Data Sci.\ Springer, Cham, 2021. \url{https://mathfordata.github.io/versions/M4D-v0.6.pdf}.
      \item Shlens, Jonathon. \textit{A tutorial on principal component
      analysis}. Preprint (2014). \url{https://arxiv.org/abs/1404.1100}.
    \end{enumerate}
\end{description}



\end{document}
